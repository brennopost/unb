\documentclass[12pt, a4paper]{article}
\usepackage{fullpage}
\usepackage[utf8]{inputenc}
\usepackage[portuguese]{babel}
\renewcommand{\baselinestretch}{1.5}

\title{Trabalho Final Empresa}
\author{Grupo Ramon}
\date{\today}

\begin{document}
\maketitle

Em 2017, a grande varejista americana Walmart anunciou que estaria retirando-se do mercado brasileiro, o qual ingressou em 1995. O motivo disso é que, embora a marca seja muito conhecida, principalmente nos EUA, o Walmart não tem tido muito sucesso em uma escala global, mesmo mantendo uma balança positivo. Em 30 de abril de 2018, a rede de mercados divulgou o seu balanço geral, que demonstrou uma queda de 30\% em relação ao ano passado e, além disso, percebe-se que o Brasil teve um faturamento local de R\$ 28 bilhões, registrando também uma queda de 4.2\%. Curiosamente, os concorrentes brasileiros do Walmart, Carrefour e Pão de Açúcar, demonstraram um crescimento de 7.2\% e 7.7\%, respectivamente.  Ainda que seja difícil afirmar quais foram os erros específicos do Walmart que levaram a essa perda nos lucros, é evidente que isso é reflexo de um mau processo de planejamento ou talvez até pela falta de um, em pelo menos um dos níveis estratégicos, táticos ou operacionais. Por conseqüência, essa deficiência no planejamento ocasionou em uma estratégia de mercado errônea por parte da empresa, reduzindo a competitividade no mercado. Por fim, há de ser considerar também uma das maiores ameaças ao mercado varejista, o e-commerce (mercado virtual); tal ameaça é apontada pelo vice-presidente executivo do Walmart, Brett Briggs, como o principal fator na redução dos lucros da empresa. No Brasil, a divisão online da rede varejista acumula aproximadamente 15 mil reclamações no site “Reclame Aqui”, cujo 80\% dos casos são resolvidos; é importante ressaltar também que a maior parte das reclamações vem de atrasos e a não-entrega do produto, além de entrega de produtos errados e no estorno do valor pago (devolução do dinheiro após o cancelamento da compra).

A fim de solucionar os problemas citados, o Walmart adota posturas diferentes: No caso brasileiro, a corporação estuda adotar uma estratégia de saída, com a liquidação de 80\% de seus negócios para a empresa gestora de capital privado Advent, essa que vive um bom momento financeiro no país e vê essa compra como uma oportunidade de expandir seu mercado. Em reação à ameaça do e-commerce, o Walmart possui uma atitude muito mais agressiva: além de firmar uma parceria com a empresa chinesa JD que já dura 2 anos, a varejista americana anunciou em março a compra 77\% da loja virtual indiana Flipkart, tornando-se a dona efetiva da empresa. A princípio, a proposta seria de US\$ 12 bilhões, mas, após quase dois anos de negociações, a compra foi fechada em US\$ 16 bilhões, demonstrando uma flexibilidade na barganha. O Walmart ainda pretende manter toda a estrutura da Flipkart, incluindo fornecedores, parcerias, etc., evidenciando uma barganha integrativa. Com essa estratégia, o Walmart espera roubar uma fatia de mercado de sua principal corrente virtual, a Amazon e, assim, transformando uma oportunidade em um ponto forte da empresa. Após as recentes derrotas financeiras, é visível uma mudança na postura do Walmart em relação ao mercado, suas oportunidades e ameaças, que agora busca retomar sua fatia de mercado e torna-se novamente competitiva com a concorrência.

\end{document}

