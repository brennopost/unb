\documentclass[11pt, a4paper]{article}
%\usepackage{fullpage}
\usepackage[a4paper, total={6in, 11in}]{geometry}
\usepackage[utf8]{inputenc}
\usepackage{kpfonts}
\usepackage[portuguese]{babel}
%\pagestyle{empty}

\title{Resenha do Capítulo 3}
\author{Brenno Cordeiro}
\date{\today}

\begin{document}

\maketitle

%    identificação da obra;
%  . apresentação da obra; 
%    descrição da estrutura;
%    descrição do conteúdo;
%    análise crítica;
%    recomendação (público-alvo);
%    identificação do autor.

BARROS, B.; PRATES, M. O Estilo Brasileiro de Administrar. São Paulo: Atlas, 1996

A obra "O Estilo Brasileiro de Administrar" demonstra a forma como os gestores brasileiros se portam frente as diferentes ações do processo administrativo. Os autores veem as empresas como um espaço sociocultural, portanto,  enfatizam a cultura como o principal agente na forma brasileira de administrar, principalmente no capítulo 3, "Impactos da Cultura Brasileira na Gestão Empresarial", foco dessa resenha, e resaltam que a existência de formas diferentes de administrar em outras culturas não torna a brasileira errada. Sendo assim, capítulo é mais uma
apreciação informativa dos impactos culturais no estilo de administrar do que uma crítica ao gestor brasileiro, como pode parecer algumas vezes ao longo do capítulo.

O Capítulo 3 é estruturado em tópicos, abordando cada processo do sistema administrativo e apresentando como os traços culturais de cada país influenciam esses processos. Os traços apresentados são, muitas vezes, repetidos em diferentes processos abordados, mostrando que esses dois estão, como os autores concluem, "intrinssicamente envolvidos". Os autores ainda utilizam de gráficos e tabelas para contribuir para a sua argumentação.

No primeiro tópico apresentado, os autores vão tratar do processo de formulação de estratégias, e como este é altamente influenciado pelas características de concentração de poder, personalismo, postura de espectador, formalismo e flexibilidade. O primeiro revela que, no Brasil, as estratégias são elaboradas nos altos escalões, como se dá na cultura norte-americana. Os autores aproveitão essa observação para fazer um \emph{link} com a segunda caraterística, mostrando que a estratégias parte, geralmente, de uma única pessoa. Assim aperece o formalismo, como uma estrutura de manutenção do poder. Assim como o personalismo demonstra quem está no poder, o formalismo cria rigidez, mantendo líderes na posição de líderes. A postura de espectador surge, principalmente, em períodos de crise, onde, diferentemente do gestores americanos e japoneses que tem uma ação ativa em relação ao seu ambiente, os gestores brasileiros tem uma postura passiva e agem defensivamente. E é também em perídos de crise que a flexibilidade se manifesta, onde buscamos "soluções pouco inovadoras porém eficientes". Como a nossa cultura é imidiatista, o planejamento estratégico apresenta dificuldades, pois focamos no curto prazo, e o planejamento tem que sempre ser recomeçado.

No segundo tópico, são abordados os impactos dos traços culturais no processo decisório. A concentração do poder se mostra presente novamente, pois não há dúvidas de quem é o resposável pelas decisões contexto brasileiro. Em uma pesquisa demonstrada pelo livro, 83\% dos gerentes entrevistados responderam afirmativamente que os gerentes devem possuir respostas para as questões postas pelo seus subordinados. Isso revela uma nova característica cultural brasileira, a do paternalismo. Vemos então o fenômeno de transferência de decisões para o cargo mais alto, criando a consciência de que os liderados são inexperientes em tomar decisões e são os líderes possuem essa capacidade. Como consequência, os superiores são vistos como \emph{superiores}. Na hora de realizar reuniões referentes a decisões, entra em ação o traço de evitar conflito, pois muitas vezes, as decisões já ocorreram, e a reunião tem apenas a função de comunicar formalmente a decisão. Por fim, um traço que influência a tomada de decisões é a impunidade, que gera grande permissividade. O processo decisório brasileiro, pode ser então explicado como "centralizado na cadeia hierárquica" e "não consultivo", como resumem os autores.


\end{document}

