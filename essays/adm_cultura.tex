\documentclass[11pt, a4paper]{article}
%\usepackage{fullpage}
\usepackage[a4paper, total={6in, 11in}]{geometry}
\usepackage[utf8]{inputenc}
\usepackage{kpfonts}
\usepackage[portuguese]{babel}
%\pagestyle{empty}

\setlength{\parskip}{\baselineskip}%
\pagenumbering{gobble}

\title{Resenha do Capítulo 3}
\author{Brenno Cordeiro}
\date{\today}

\begin{document}

%\maketitle

{\Large Resenha do capítulo 3}

{\large Brenno Cordeiro}

%    identificação da obra;
%  . apresentação da obra; 
%    descrição da estrutura;
%    descrição do conteúdo;
%    análise crítica;
%    recomendação (público-alvo);
%    identificação do autor.

BARROS, B.; PRATES, M. O Estilo Brasileiro de Administrar. São Paulo: Atlas, 1996

A obra ``O Estilo Brasileiro de Administrar'' demonstra a forma como os gestores brasileiros se portam frente as diferentes ações do processo administrativo. Os autores veem as empresas como um espaço sociocultural, portanto,  enfatizam a cultura como um dos principais agentes na forma brasileira de administrar, principalmente no capítulo 3, ``Impactos da Cultura Brasileira na Gestão Empresarial'', foco dessa resenha, e resaltam que a existência de formas diferentes de administrar em outras culturas não torna a brasileira errada. Sendo assim, capítulo é mais uma
apreciação informativa dos impactos culturais no estilo de administrar do que uma crítica ao gestor brasileiro, como pode parecer algumas vezes ao longo do capítulo.

O Capítulo 3 é estruturado em tópicos, abordando cada processo do sistema administrativo e apresentando como os traços culturais de cada país influenciam esses processos. Os traços apresentados são, muitas vezes, repetidos em diferentes processos abordados, mostrando que esses dois estão, como os autores concluem, ``intrinssicamente envolvidos''. Os autores ainda utilizam de gráficos e tabelas para contribuir para a sua argumentação.

No primeiro tópico apresentado, os autores vão tratar do processo de formulação de estratégias, e como este é altamente influenciado pelas características de concentração de poder, personalismo, postura de espectador, formalismo e flexibilidade. O primeiro revela que, no Brasil, as estratégias são elaboradas nos altos escalões, como se dá na cultura norte-americana. Os autores aproveitão essa observação para fazer um \emph{link} com a segunda caraterística, mostrando que a estratégias parte, geralmente, de uma única pessoa. Assim aperece o formalismo, como uma estrutura de manutenção do poder. Assim como o personalismo demonstra quem está no poder, o formalismo cria rigidez, mantendo líderes na posição de líderes. A postura de espectador surge, principalmente, em períodos de crise, onde, diferentemente do gestores americanos e japoneses que tem uma ação ativa em relação ao seu ambiente, os gestores brasileiros tem uma postura passiva e agem defensivamente. E é também em perídos de crise que a flexibilidade se manifesta, onde buscamos ``soluções pouco inovadoras porém eficientes''. Como a nossa cultura é imidiatista, o planejamento estratégico apresenta dificuldades, pois focamos no curto prazo, e o planejamento tem que sempre ser recomeçado.

No segundo tópico, são abordados os impactos dos traços culturais no processo decisório. A concentração do poder se mostra presente novamente, pois não há dúvidas de quem é o resposável pelas decisões contexto brasileiro. Em uma pesquisa demonstrada pelo livro, 83\% dos gerentes entrevistados responderam afirmativamente que os gerentes devem possuir respostas para as questões postas pelo seus subordinados. Isso revela uma nova característica cultural brasileira, a do paternalismo. Vemos então o fenômeno de transferência de decisões para o cargo mais alto, criando a consciência de que os liderados são inexperientes em tomar decisões e são os líderes possuem essa capacidade. Como consequência, os superiores são vistos como \emph{superiores}. Na hora de realizar reuniões referentes a decisões, entra em ação o traço de evitar conflito, pois muitas vezes, as decisões já ocorreram, e a reunião tem apenas a função de comunicar formalmente a decisão. Por fim, um traço que influência a tomada de decisões é a impunidade, que gera grande permissividade. O processo decisório brasileiro, pode ser então explicado como ``centralizado na cadeia hierárquica'' e ``não consultivo'', como resumem os autores.

O tópico 3 trata sobre o processo de liderança. Já nos primeiros parágrafos nos é apresentada uma pesquisa que mostra que a percepção da maioria dos gerentes brasileiros quanto a motivação dos executivos vem da conquista do poder mais do que atingir resultados. Esse comportamento é parcido na França e na Itália, mas difernte nos Estados Unidos e na Alemanha. A tradição familiar também se mostra influente na liderança, onde o líder oferece proteção e os liderados retribuem o favor. É comum ainda, nas empresas brasileiras, um líder se destacar, muitas vezes se fundindo com a empresa, passando a ser representada pela imagem de seu princial gerente. Assim, os objetivos pessoais e os objetivos da organizam devem estar de acordo para garantir a qualidade, e o líder terá o papel de legitimar os objetivos. Isso se dá, em nossa cultura, pela lealdade pessoal, traço que influencia diversos processos. De modo geral, a estrutura paternalista se apresenta em todos os níveis, e vemos uma relação de poder baseada na hierarquia e na lealdade pessoal.

O quarto tópico aborda o processo de coesão organizacional. Nas empresas, é importante manter ``seus componentes unidos em torno do mesmo objetivo'', como definem os autores. Novamente, o traço da lealdade pessoal se mostra como ator no processo. Desta forma, as relações pessoais são o principal fator de coesão social nas empresas brasileiras. A empresa tenta criar um ambiente familiar, da forma que passa a ser sua segunda casa, buscando fazer com que os empregados \emph{pertençam} àquele ambiente. Porém, essas relações devem ser limitadas por códigos e protocolos rígidos. Outro ponto do processo de coesão organizacional brasileiro é a resolução de conflitos. Muitas vezes, os conflitos são identificados, mas sua solução não é direta, e algumas vezes é adiada, por causa do nosso instinto de evitar conflitos. Quando surgem problemas de coordenação de grupos nas empresas, surge a alternativa da estrutura organizacional. No Brasil, esta será influenciada pela concentração de poder e pelo personalismo, criando uma estrutura piramidal e hierárquica. O formalismo aparece para acerscentar burocracia ao bolo.

No quinto tópico conhecemos os efeitos da cultura no processo de inovação e mudança. Como fazemos parte de uma cultura de ``ser'', é sempre mais fácil contornar as limitações com soluçõescriativas, preservando o núcleo de valores, do que lidar com mundanças. Isso ocorre por causa do nosso traço de evitar conflitos. E a nossa postura de espectador nos cria uma rejeição ao risco. Apenas inovamos em situações de crise, e vencemos os obstáculos na base do improviso. Os autores não afirma que o brasileiro não sabe inovar. Na verdade, mostram que as ideais inovadoras existem, mas como não há incentivo por parte da empresa, não passam de ideias. O traço da impunidade volta a aparecer nos seus dois sentidos: premiação e punição. No Brasil, há baixo \emph{feedback}, e faltam estímulos à mudança, por causa da impunidade. A mudança, para as empresas brasileiras, portanto, é geralmente um fator corretivo, minuscioso e vem de estímulos externos, geralmente uma crise.

O tópico 6 informa sobre o processo de motivação. Segundo os autores, o clima de baixa motivação nas empresas brasileiras pode ser explicado pelos traços de impunidade e desigualdade no trabalho. Os líderes, entretanto, tentam criar incentivos através da particiação. A coesão social, feita pelo personalismo, também pode ser motivacional ao fazer os funcionários se sentirem parte do grupo. Nas culturas japonesas e alemães, a motivação se da pela segurança individual. Na cultura americana, são mais fatores que compõe o ideal motivacional, como o sucesso pessoal e a auto-realização. A cultura parternalista apresenta sua parcela de influência no processo de motivação, que faz com que os funcionários acreditem que é dever da empresa fornecer treinamento. A empresa, por sua vez, não o oferece pois há baixa lealdade. 

O capítulo termina com a conclusão dos autores, que mostram que os traços culturais concordam, e não concorrem, na produção do estilo brasileiro de administrar. Isso não é o suficiente, porém, para acreditar que seja impossível romper a relação entre essas características. É possível, na verdade, que uma mudança em um deles afete toda a cadeia dos processos. Os autores classificam como irrealista ideais que tentam unir traços de diferentes culturas em uma só administração. Os núcleos culturais são únicos, e os processos devem ser fortemente analisados antes de se tentar importá-los de outros países, focando principalmente nas traços culturais que os envolvem.

\end{document}

